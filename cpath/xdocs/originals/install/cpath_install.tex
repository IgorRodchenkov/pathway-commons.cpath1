\documentclass{article}
\usepackage{graphicx}
\title{cPath:  Installation Guide}
\author{Ethan Cerami, Memorial Sloan-Kettering Cancer Center (MSKCC)}
\date{January 16, 2007}
\begin{document}


\maketitle

\section{About this Document}

This document describes how to build a local installation of cPath, administer cPath via the command line, import BioPAX and identifier mapping files, and deploy the cPath web application.  This document refers only to the latest version of cPath (Version 0.7 Beta, from February, 2007).  For questions regarding this document or questions regarding cPath installation, please send email to cerami AT cbio.mskcc.org.

\begin{figure}[htbp]
  \centering
  \fbox{
    \includegraphics{icon.png}
  }
  \caption{This is the Reactome logo}
  \label{Reactome}
\end{figure}

\section{Introduction to cPath}

cPath is an open source pathway database and software suite designed for systems biology research. The main features of cPath include:

\begin{itemize}

  \item built-in support for the Biological Pathways Exchange (BioPAX) XML format.

  \item built-in identifier mapping service for linking identical interactors and linking to external resources.

  \item configurable web interface for browsing and searching biological pathways.

  \item simple HTTP URL based XML web service.

\end{itemize}

cPath is currently being developed by the Computational Biology Center at Memorial Sloan-Kettering Cancer Center, and the University of Toronto.

\bigskip

For a complete description of cPath software, please refer to:

\begin{itemize}
\item Ethan G Cerami, Gary D Bader, Benjamin E Gross, and Chris Sander, \textbf{cPath: open source software for collecting, storing, and querying biological pathways} BMC Bioinformatics 2006, 7:497 [doi:10.1186/1471-2105-7-497]
http://www.biomedcentral.com/1471-2105/7/497/abstract
\end{itemize}

\section{System Requirements}

cPath can be installed on Linux, Windows or Mac OS X.  However, before installing cPath, you must have the following installed: 

\begin{itemize}

\item Java 1.5:  Information available at:  http://java.sun.com/j2se/1.5.0/

\item MySQL: Information available at: http://www.mysql.com/.  cPath has been 
tested with MySQL 4.0 and 4.1. 

\item Apache Tomcat Server, 4.1X or above: Information available at: 
http://jakarta.apache.org/tomcat/.  cPath has been tested with Tomcat 4.1 
and Tomcat 5.0. 
 
\item Apache Ant 1.6 (or later): Information available at: http://ant.apache.org/. 
 
\item Perl 5.0 (or later): Information available at: http://www.perl.org/. 

% What version of MySQL am I currently running @ work?
% What version of Apache Tomcat Server are we running on toro, cbio and local?

\end {itemize}

\section{cPath Installation:  Step-By-Step Instructions }

\subsection{Install required software}

If you have not already done so, install Java 1.5, MySQL, Apache Tomcat, and Ant locally. Once you have verified that the installations are working correctly, proceed to the next step. 

\subsection{Install Third-Party Libraries Required by Ant}

The ant build.xml file requires a few third-party libraries.  Copy all the .jar files in \verb+[CPATH_HOME]/lib/ant+ to 
\verb+[ANT_HOME]/lib+.  For example: 
 
% What happens if somehow doesn't even know what ant is?
% And, why do we need them to use ant?  Couldn't this be much simplified?

\bigskip

\texttt{cp lib/ant/*.jar /Users/smith/libraries/apache-ant-1.6.1/lib}

\subsection{(Optional) Update your build.properties file}

Here is an excerpt from the build.properties file: 
 
\begin{verbatim}
# Application Name and Version 
app.name=cpath 
app.version=0.5 
 
# Database Settings 
db.name=cpath 
db.user=tomcat 
db.password=kitty 
db.host=localhost 
\end{verbatim}

% The example above is very old.  Need to replace with new one.
% Also, need a short description of what build.properties is and why one would even need to modify it.


% Add a new section on how to configure the web interface front end.


\end{document}             % End of document.
