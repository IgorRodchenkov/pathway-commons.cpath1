\documentclass[letterpaper,12pt]{article}
\pdfpagewidth 8.5in
\pdfpageheight 11in 
\usepackage{graphicx}
\title{cPath:  Installation Guide}
\author{cPath Development Team}
\date{February 15, 2007}
\begin{document}

\maketitle

\tableofcontents

\section{About this Document}

This document describes how to build a local installation of cPath, administer cPath via the command line, import BioPAX and identifier mapping files, and deploy the cPath web application.  This document refers only to the latest version of cPath (Version 0.7 Beta, from February, 2007).  For questions regarding this document or questions regarding cPath installation, please send email to cerami AT cbio.mskcc.org.

\section{Introduction to cPath}

cPath is an open source pathway database and software suite designed for systems biology research. The main features of cPath include:

\begin{itemize}

  \item built-in support for the Biological Pathways Exchange (BioPAX) XML format.

  \item built-in identifier mapping service for linking identical interactors and linking to external resources.

  \item configurable web interface for browsing and searching biological pathways.

  \item simple HTTP URL based XML web service.

\end{itemize}

cPath is currently being developed by the Computational Biology Center at Memorial Sloan-Kettering Cancer Center, and the University of Toronto.

\bigskip

For a complete description of cPath software, please refer to:

\begin{itemize}
\item Ethan G Cerami, Gary D Bader, Benjamin E Gross, and Chris Sander, \textbf{cPath: open source software for collecting, storing, and querying biological pathways} BMC Bioinformatics 2006, 7:497 [doi:10.1186/1471-2105-7-497]
http://www.biomedcentral.com/1471-2105/7/497/abstract
\end{itemize}

\section{System Requirements}

cPath can be installed on Linux, Windows or Mac OS X.  However, before installing cPath, you must have the following installed: 

\begin{itemize}

\item Java 1.5:  Information available at:  http://java.sun.com/j2se/1.5.0/

\item MySQL: Information available at: http://www.mysql.com/.  cPath has been 
tested with MySQL 4.0 and 4.1. 

\item Apache Tomcat Server, 4.1X or above: Information available at: \linebreak
http://jakarta.apache.org/tomcat/.  cPath has been tested with Tomcat 4.1 
and Tomcat 5.0. 
 
\item Apache Ant 1.6 (or later): Apache Ant is a software tool for automating software build processes.  Information is available at: http://ant.apache.org/. 
 
\item Perl 5.0 (or later): Information available at: http://www.perl.org/. 

\end {itemize}

\section{cPath Installation:  Step-By-Step Instructions }

This section provides complete step-by-step instructions to building a local installation of cPath.

\subsection{Getting Started}

\subsubsection{Install required software}

If you have not already done so, install Java 1.5, MySQL, Apache Tomcat, Ant and Perl locally. Once you have verified that the installations are working correctly, proceed to the next step. 

\subsubsection{Download a cPath Source Distribution}

The cPath source is currently available for download at: 

http://cbio.mskcc.org/\verb+dev_site+/cpath/source.html.

\bigskip

We currently provide two download options:

\begin{itemize}

\item the current stable release:  this is the current stable release (version 0.7), now used in production;  and

\item the nightly snapshot:  this is the latest source code from our CVS repository.  This contains the very latest features, but please use at your own risk.

\end{itemize}

Once you have downloaded a source distribution, unzip it, and place it anywhere on your hard drive.  

\subsubsection{Set a CPATH HOME Variable}

Next, you will need to set a \verb+CPATH_HOME+ environment variable. This variable must point to the cPath distribution that you obtained in the previous step.  If you are using Mac OS X or Linux, you can do so via the shell. For example, here is an excerpt from a \verb+.bash_login+ file:

\begin{verbatim}
# Set up CPATH_HOME Variable
export CPATH_HOME=/Users/cerami/dev/sander/cpath
\end{verbatim}

\subsubsection{Configure your settings via the build.properties file}

The root cpath directory contains a configuration file named build.properties.  Before proceeding, you must configure the the database user/password properties in build.properties to match your current MySQL installation.  For example, you must set db.user to an existing MySQL user, and this user must have the rights to create new tables.  If you are not sure how to set up new MySQL users, and you are eager to get started, try setting db.name=root and set the appropriate password.  
 
Here is an excerpt from build.properties:

\begin{verbatim}
# Application Name and Version
app.name=cpath
app.version=0.7

# Database Settings
db.name=cpath
db.user=tomcat
db.password=kitty
db.host=localhost 
\end{verbatim}

Complete instructions on adding new MySQL users is available online at:
\verb+http://www.mysql.com/doc/en/Adding_users.html+.

\subsubsection{Create the cPath Database in MySQL}

Next, create the cPath database and tables.  To do so, run the initDb.pl script in the bin directory. For example:

\begin{verbatim}
> cd bin
> ./initDb.pl

Using CPATH_HOME /Users/cerami/dev/sander/cpath
Script to Initialize cPath Database
===================================

Using Database Name:      cpath
Using Database Host:      localhost
Using Database User:      tomcat
Using Database Password:  kitty

! Running this command will delete all existing data !
Are you sure you wish to proceed (Type: YES):  YES
Importing MySQL File:  cpath.sql
Importing MySQL File:  seed.sql
Importing MySQL File:  reference.sql
Done.  cPath is now ready to go...
\end{verbatim}

\subsection{Building cPath from Source}

Once the cPath database is set up, the next step is to compile and build cPath from source.

\subsubsection{Install Third-Party Libraries Required by Ant}

cPath uses Apache Ant to automate the process of compilation and deployment.  However, the default cPath Ant build.xml file requires a few third-party libraries.  To install these libraries, simply copy all the .jar files in \verb+[CPATH_HOME]/lib/ant+ to \verb+[ANT_HOME]/lib+.  For example, on Mac OS X or Linux, type the following command:
 
\bigskip

\texttt{cp lib/ant/*.jar /Users/smith/libraries/apache-ant-1.6.1/lib}

\subsubsection{Compile cPath from Source}

To compile cPath, move to the root cPath directory and type:

\begin{verbatim}
> ant
\end{verbatim}

\subsection{Deploying the cPath Web Application}

If you have successfully followed all the steps above, you will now have an instance of cPath without any data.  As the next step, you will probably want to import some data.  But, if you are eager to deploy the cPath web application, follow the steps below:

\subsubsection{Generate the cPath WAR File}  
\label{generate-war}

The cPath WAR (Web Application Archive) contains the complete contents of cPath, ready for deployment to Tomcat. To generate the war file, move to the root cPath directory and type:

\begin{verbatim}
> ant war
\end{verbatim}

\subsubsection{Deploy cpath.war to Tomcat}

To do so, simply copy the cpath WAR file from build/war to your Tomcat webapps directory. For example:
\begin{verbatim}
cp build/war/cpath_0.5.war+ ~/libraries/jakarta-tomcat-5.0.28/webapps/
\end{verbatim}

Copying your war file here will cause Tomcat to automatically load and start cPath.

\subsubsection{Verify installation}

Open a web browser, and go to: http://localhost:8080/cpath/index.jsp. Verify that you can see the cPath Home Page.

\section{Importing and Indexing BioPAX Data}

Once you have installed cPath, your next step is to import data.  To do so, you use the command line administrator tool located in [\verb+CPATH_HOME+]/bin.

\subsection{Import Meta-Data Details}

The first step is to load meta-data regarding the source of your pathway data.  This meta-data contains details about the source database, including the name of the database, a short description, home page URL, Etc.  For example, if you are loading a BioPAX file from HumanCyc, you must first load meta-data regarding HumanCyc into cPath.  The meta-data is very important, as it provides a means to link back to the original data provider, and give them due credit.

To create a meta-data file, you must first create a valid XML configuration file, and then explicitly load this file into cPath.  A sample meta-data file is found in \verb+dbData/externalDb/pathway_commons.xml+.  This file loads all meta-data for all sources currently found at pathwaycommons.org.

Below is an excerpt:

\begin{verbatim}
<?xml version="1.0" encoding="UTF-8"?>
<!DOCTYPE external_database_list SYSTEM "external_db.dtd">
<external_database_list>
    <external_database type="INTERACTION_PATHWAY_UNIFICATION">
        <name>Reactome</name>
        <description>Reactome: a knowledgebase of...</description>
        <home_page_url>http://reactome.org</home_page_url>
        <entity_url>
            <url_pattern>http://reactome.org/cgi-bin/...</url_pattern>
            <sample_id>12345</sample_id>
        </entity_url>
        <icon_path>Reactome.png</icon_path>
        <controlled_terms>
            <master_term>REACTOME</master_term>
        </controlled_terms>
    </external_database>
    ...
\end{verbatim}

Using this as a sample, you can probably figure out how to create your own meta-data configuration files.  Note that you can also include a database icon by specifying the \verb+icon_path+ element.  For best results, the database icons should be approximately 66 x 66 pixels large.

Once you have created a meta-data file, move to \verb+[CPATH_HOME]\bin+, and load the file via the admin.pl script.  For example:

\begin{verbatim}
./admin.pl -f ../dbData/externalDb/pathway_commons.xml import
\end{verbatim}

In the above example, the -f flag specifies the location of the meta-data file, and the import command indicates that you wish to import new data.

\subsection{Import BioPAX Files}

Next, import your BioPAX files.  For example, the following command loads the Reactome Apoptosis pathway:

\begin{verbatim}
./admin.pl -f ../testData/biopax/Apoptosis.owl import
\end{verbatim}

% Add details re:  warning message.

\subsection{Fetch PubMed Details}

Once you have imported all your BioPAX files, the next step is to run the PubMed Fetcher.  If a BioPAX record contains a PubMedID, but does not contain the complete citation information regarding the article, the fetcher will automatically retrieve this data from PubMed.  To run the fetcher, type:

\begin{verbatim}
./admin.pl pop_ref
\end{verbatim}

Note that NCBI requires a 3 second delay between batch requests.  So, downloading all PubMed records may take some time.

\subsection{Index cPath Records}

To enable the cPath search engine, you must index all cPath records and create a full-text index.  To do so, run the index command.  For example:

\begin{verbatim}
./admin.pl -f index
\end{verbatim}

Remember to re-index cPath after importing any new data.

\subsection{Regenerate the WAR File}

Once all your data has been imported and indexed, you must re-generate the WAR file and redeploy to Tomcat.  See \ref{generate-war} above.

\section{Configuring cPath [Under Construction]}

[Under Construction]

\subsection{Configuring the Look and Feel of cPath}

% Add details re:  purging the cache.

\subsection{Configuring Logging}

\section{Known Issues [Under Construction]}

[Under Construction]
\subsection{Known Issue: Packet Too Big Exception}
\subsection{Known Issue: Out of Memory Error}

\end{document}             % End of document.

% What version(s) of PSI-MI does cPath support?
